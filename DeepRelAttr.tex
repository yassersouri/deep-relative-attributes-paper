\documentclass[10pt,twocolumn,letterpaper]{article}

\usepackage{cvpr}
\usepackage{times}
\usepackage{epsfig}
\usepackage{graphicx}
\usepackage{amsmath}
\usepackage{amssymb}

\usepackage{subfigure}
\usepackage{algorithm}
\usepackage{algorithmic}
\usepackage{mathrsfs}
\usepackage{soul}
\usepackage{multirow}
\usepackage{comment}

%\usepackage[sort&compress,square,comma]{natbib}

\usepackage{pgfplots}
\usepackage{pgfplotstable}
\usepackage{filecontents,pgfplots}
\usepackage{tikz}
\usetikzlibrary{shapes,matrix,arrows,decorations.pathmorphing}
\usepackage{sci}
\renewcommand{\algorithmicrequire}{\textbf{Input:}}
\renewcommand{\algorithmicensure}{\textbf{Output:}}


% Include other packages here, before hyperref.

% If you comment hyperref and then uncomment it, you should delete
% egpaper.aux before re-running latex.  (Or just hit 'q' on the first latex
% run, let it finish, and you should be clear).
\usepackage[pagebackref=true,breaklinks=true,letterpaper=true,colorlinks,bookmarks=false]{hyperref}

%\cvprfinalcopy % *** Uncomment this line for the final submission

\def\cvprPaperID{725} % *** Enter the CVPR Paper ID here
\def\httilde{\mbox{\tt\raisebox{-.5ex}{\symbol{126}}}}

% Pages are numbered in submission mode, and unnumbered in camera-ready
\ifcvprfinal\pagestyle{empty}\fi
\begin{document}

%%%%%%%%% TITLE
\title{Deep Relative Attributes}

\author{First Author\\
Institution1\\
Institution1 address\\
{\tt\small firstauthor@i1.org}
% For a paper whose authors are all at the same institution,
% omit the following lines up until the closing ``}''.
% Additional authors and addresses can be added with ``\and'',
% just like the second author.
% To save space, use either the email address or home page, not both
\and
Second Author\\
Institution2\\
First line of institution2 address\\
{\tt\small secondauthor@i2.org}
}

\maketitle
%\thispagestyle{empty}

%%%%%%%%% ABSTRACT
\begin{abstract}
Relative Attributes are a very natural way of thinking in terms of attributes and communicating with machines. The idea was introduced in the award winning ICCV 2011 paper by D. Parikh and K. Grauman \cite{parikh2011}. In this project we want to improve their system by using a Deep Neural Network instead of a RankSVM to do the ranking. This way we can also use Convolutional Layers to learn the features end-to-end or fine-tune the features.
\end{abstract}



% !TeX root=DeepRelAttr.tex
% !TEX TS-program = pdfLatex

%%%%%%%%%%%%%%%%%%%%%% INTRODUCTION %%%%%%%%%%%%%%%%%%%%%%%%%%%%%%
\section{Introduction}

 After the "Relative Attributes" paper \cite{parikh2011}, there was a stream of papers that aimed to solve the same or similar task (\cite{Li2013,Yu2014,Sandeep_2014_CVPR,Lee_2013_7540}) using a different and often more complex model instead of the original RankSVM model. I think as progress in the "Visual Recognition" field has shown, for solving the problem you actually need to change the features not the model. So in this project I actually want to experiment with how learning the features end-to-end (or fine-tuning the features) can improve Relative Attributes accuracy and power. 


% TODO: Add ICCV 2015 and BMVC 2015 papers
%%%%%%%%%%%%%%%%%%%%%%%%%% RELATED WORKS %%%%%%%%%%%%%%%%%%%%%%%%%
\section{Related Works}
\label{sec.2}

We usually describe visual concepts with their attributes. %, and how they look. 
Attributes are, therefore, mid-level representations for describing objects and scenes. In an early work on attributes, Farhadi \etal~\cite{Farhadi09describingobjects} proposed to describe objects using mid-level attributes. In another work \cite{Farhadi2010EveryPT}, the authors described images based on %their attributes, basically 
a semantic triple ``object, action, scene". In the recent years, attributes have shown great performance in object recognition \cite{Farhadi09describingobjects,7298613}, action recognition \cite{6838985,5995353} and event detection \cite{6475038}. Lampert \etal~\cite{6571196} predicted unseen objects using a zero-shot learning framework, incorporating the binary attribute representation of the objects. %in which binary attribute representation of the objects were incorporated. 

Although detection and recognition based on the presence of attributes appeared to be quite interesting, comparing attributes enables us to easily and reliably search through high-level data derived from \eg, documents or images. For instance, Kovashka \etal~\cite{KovashkaG13} proposed a relevance feedback strategy for image search using attributes and their comparisons. In order to establish the capacity for comparing attributes, we need to move from binary attributes towards describing attributes relatively. In the recent years, relative attributes have attracted the attention of many researchers.
%, in which a function is learned for each attribute that enables comparison between different attributes.
For instance, a linear relative comparison function is learned in \cite{parikh2011}, based on RankSVM \cite{Joachims2002} and a non-linear strategy in \cite{Li2012RelativeFF}. In another work, Datta \etal~\cite{5771429} used trained rankers for each facial image feature and formed a global ranking function for attributes.

For the process of learning the attributes, different types of low-level image features are often  incorporated. For instance, Parikh and Grauman~\cite{parikh2011} used 512-dimensional GIST \cite{Aude01} descriptors as image features, while Jayaraman \etal~\cite{6909607} used histograms of image features, and reduced their dimensionality using PCA. Other works tried learning attributes through \eg, local learning \cite{1641014} or fine-grained comparisons \cite{Yu2014}. Yu and Grauman \cite{Yu2014} proposed a local learning-to-rank framework for fine-grained visual comparisons, in which the ranking model is learned using only analogous training comparisons. In another work \cite{Yu2015}, they proposed a local Bayesian model to rank images, which are hardly distinguishable for a given attribute. However, none of these methods leverage the effectiveness of feature learning methods and only use engineered and hand-crafted features for predicting relative attributes. %did to leverage the effectiveness of convolutional neural networks and feature learning for relative attribute prediction.

As could be inferred from the literature, it is very hard to decide what low-level image features to use for identifying and comparing visual attributes. Recent studies show that features learned through the convolutional neural networks (CNNs) \cite{LeCun1989HandwrittenDR} (also known as deep features) could achieve great performance for image classification \cite{Krizhevsky2012ImageNetCW} and object detection \cite{6909475}. Zhang \etal~\cite{6909608} utilized CNNs for classifying binary attributes. In other works, Escorcia \etal~\cite{Escorcia_2015_CVPR} proposed CCNs with attribute centric nodes within the network for establishing the relationships between visual attributes. Shankar \etal~\cite{Shankar_2015_CVPR} proposed a weakly supervised setting on convolutional neural networks, applied for attribute detection. Khan \etal~\cite{khan15} used deep features for describing human attributes and thereafter for action recognition, and Huang \etal~\cite{Huang_2015_ICCV} used deep features for cross-domain image retrieval based on binary attributes.  

Neural networks have also been extended for learning-to-rank applications. One of the earliest networks for ranking was proposed by Burges\etal~\cite{Burges2005}, known as RankNet. The underlying model in RankNet maps an input feature vector to a Real number. The model is trained  by presenting the network pairs of input training feature vectors with differing labels. Then, based on how they should be ranked, the underlying model parameters are updated. This model is used in different fields for ranking and retrieval applications, \eg, for personalized search \cite{song2014} or content-based image retrieval \cite{Wan2014}. In another work, Yao \etal~\cite{YaoCVPR2016} proposed a ranking framework for videos for first-person
video summarization, through recognizing video highlights. They incorporated both spatial and temporal streams through 2D and 3D CNNs and detect the video highlights.



% !TeX root=DeepRelAttr.tex
% !TEX TS-program = pdfLatex

%%%%%%%%%%%%%%%%%%%%%%% END-TO-END DEEP RELATIVE ATTRIBUTES %%%%%%%%%%%%%%%
\section{Proposed Method}
\label{sec.3}


%%%%%%%%%%%%%%%%%%%%%%%% Figure 2 %%%%%%%%%%%%%%%%%%%%%%%%%%%%%%%%%%%%%%%%%%%%%%%%%%%%%
\begin{figure*}
\label{fig.2}
\centering
\scalebox{.3}
{
% We need layers to draw the block diagram
\pgfdeclarelayer{background}
\pgfdeclarelayer{foreground}
\pgfsetlayers{background,main,foreground}

% Define a few styles and constants
\tikzstyle{sensor}=[draw, fill=blue!20, text width=5em, 
    text centered, minimum height=2.5em]
\tikzstyle{ann} = [above, text width=5em]
\tikzstyle{naveqs} = [sensor, text width=6em, fill=red!20, 
    minimum height=12em, rounded corners]
\def\blockdist{2.3}
\def\edgedist{2.5}

\begin{tikzpicture}
	% feature extraction part rectangle
	\node [scale=3] (testtitle) at (10.5cm, 5.0cm) {\textcolor{red}{Test time}};
	\draw [rounded corners=1cm, dashed, line width=3, red] (-3.5cm, 4cm) rectangle (24.5cm,-2.5cm);
	% images
	\node (im1) at (0cm,1cm) [draw] {\includegraphics[scale=1]{im1.jpg}};
	\node (im2) at (0cm, -6cm) [draw] {\includegraphics[scale=1]{im2.jpg}};

	% topconv1 layer
	\node (tconv1) at (5.1cm, 1cm) {};
%	\draw [fill=blue!20] (5.6cm,3.1cm) rectangle (9.6cm,0.1cm);
	\draw [fill=blue!20] (5.4cm,2.9cm) rectangle (9.4cm,-0.1cm);
	\draw [fill=blue!20] (5.2cm,2.7cm) rectangle (9.2cm,-0.3cm);
	\draw [fill=blue!20] (5cm,2.5cm) rectangle (9cm,-0.5cm);

	% bottomconv1 layer
	\node (bconv1) at (5.1cm, -6cm) {};
%	\draw [fill=blue!20] (5.6cm,-3.9cm) rectangle (9.6cm,-6.9cm);
	\draw [fill=blue!20] (5.4cm,-4.1cm) rectangle (9.4cm,-7.1cm);
	\draw [fill=blue!20] (5.2cm,-4.3cm) rectangle (9.2cm,-7.3cm);
	\draw [fill=blue!20] (5cm,-4.5cm) rectangle (9cm,-7.5cm);

	% arrows from images to conv1s
	\path [draw, ->, line width=3] (im1.east) -- node [above, scale=3] {$I_i$} (tconv1);
	\path [draw, ->, line width=3] (im2.east) -- node [above, scale=3] {$I_j$} (bconv1);

	% topconv2 layer
	\node (tconv2) at (12cm, 1cm) {};
	\draw [fill=blue!20] (12.6cm,3.1cm) rectangle (16.6cm,0.1cm);
	\draw [fill=blue!20] (12.4cm,2.9cm) rectangle (16.4cm,-0.1cm);
	\draw [fill=blue!20] (12.2cm,2.7cm) rectangle (16.2cm,-0.3cm);
	\draw [fill=blue!20] (12cm,2.5cm) rectangle (16cm,-0.5cm);

	% bottomconv2 layer
	\node (bconv2) at (12cm, -6cm) {};
	\draw [fill=blue!20] (12.6cm,-3.9cm) rectangle (16.6cm,-6.9cm);
	\draw [fill=blue!20] (12.4cm,-4.1cm) rectangle (16.4cm,-7.1cm);
	\draw [fill=blue!20] (12.2cm,-4.3cm) rectangle (16.2cm,-7.3cm);
	\draw [fill=blue!20] (12cm,-4.5cm) rectangle (16cm,-7.5cm);

	\path (tconv1.east)+(4.65cm, 0cm) -- node [scale=4]{\dots} (tconv2);
	\path (bconv1.east)+(4.65cm, 0cm) --node [scale=4]{\dots} (bconv2);

	\node (convnet) [below, scale=3.5] at (11cm, -8cm) {ConvNet};

	% toprank
	\node (tconvout) at (16.6cm, 1cm) {};
	\node (trank) at (20.5cm, 1cm) [rectangle, draw, fill=red!20, scale=3, align=center] {Ranking \\  Layer};
	\path [draw, line width=3, ->] (tconvout) -- node[above, scale=3] {$\psi_i$} (trank.west);

	% bottomrank
	\node (bconvout) at (16.6cm, -6cm) {};
	\node (brank) at (20.5cm, -6cm) [rectangle, draw, fill=red!20, scale=3, align=center] {Ranking \\  Layer};
	\path [draw, line width=3, ->] (bconvout) -- node[above, scale=3] {$\psi_j$} (brank.west);

	% shared rank layer
	\node (dashcenter) at (10.75cm, 0cm) {};
	\path [draw, line width=2, <->] (trank.south) -- node [scale=2, draw, rectangle, fill=white, line width=0] {\rotatebox{90}{shared}} (brank.north);
	\path [draw, line width=2, <->] (trank.south -| dashcenter) -- node [scale=2, draw, rectangle, fill=white, line width=0] {\rotatebox{90}{shared}} (brank.north -| dashcenter);

	% posterior
	\node (posterior) at (26cm, -2.5cm) [rectangle, draw, fill=orange!10, scale=3, minimum width=15] {\rotatebox{90}{Posterior}};

	% connect rank layer to posterior
	\path [draw, line width=2, ->] (trank.east) -- node [scale=3, above] {$r_i$} (posterior);
	\path [draw, line width=2, ->] (brank.east) -- node [scale=3, above] {$r_j$} (posterior);

	% BXE
	\node (bxent) at (33cm, -8cm) [draw,fill=green!10, regular polygon, regular polygon sides=3, shape border rotate=-90, scale=8, line width=2] {};
	\node at (33.2cm, -8cm) {\Huge BXEnt};

	% target value
	\node (target) at (26cm , -9.5cm) [scale=3] {target};

	% connect to bxe
	\draw [line width=3, ->] (posterior.east) -- ++(2cm, 0cm) |- node [above right, scale=3] {$p_{ij}$}  (bxent.north west);
	\draw [line width=3, ->] (target.east) -- node [below, scale=3] {$t_{ij}$}  (bxent.south west |- target.east);

	% loss node
	\node (loss) [scale=3] at (40cm, -8cm) { loss};

	% connect bxent to loss
	\draw [line width=3, ->] (bxent.east) --  (loss.west);
\end{tikzpicture}}
\end{figure*}
%%%%%%%%%%%%%%%%%%%%%%%% Figure 2 %%%%%%%%%%%%%%%%%%%%%%%%%%%%%%%%%%%%%%%%%%%%%%%%%%%%%

We propose to use a ConvNet based deep neural network that is trained to optimize the appropriate ranking loss for the task of predicting relative attribute strength. The network architecture consists of two parts, the \textit{feature extraction part} and the \textit{ranking part}.

The feature extraction part takes as input an image $I_i$ and outputs the feature representation of that image $\psi_i \in \mathbb{R}^d$ (Figure \cite{fig.2}).
As feature extraction part, some intermediate layer of any ConvNet architecture such as AlexNet \cite{krizhevski}, VGGNet \cite{verydeep} or GoogleNet \cite{googlenet}, can be used.

% In our experiments we use the fc7 (the last layer before the probability layer) of VGG-16 \cite{verydeep} as the feature extraction part of the network.

The ranking part, maps the feature representation $\psi_i$ to the attribute strength or absolute rank of the input $r_i \in \mathbb{R}$. Then given two attribute strengths generates the posterior $P_{ij}$ that the strength of the attribute in $I_i$ is larger than the strength of the attribute in $I_j$. 

\subsection{Learning to Rank Using Gradient Descent}


\subsection{Deep Relative Attributes}



% !TeX root=DeepRelAttr.tex
% !TEX TS-program = pdfLatex

%%%%%%%%%%%%%%%%%%%%% EXPERIMENTS %%%%%%%%%%%%%%%%%%%%%%%%%%%%%%%%%%%%%
\section{Experiments}

\subsection{Datasets}
To assess the performance of the proposed method, we have evaluated our method on five datasets.\;\textbf{PubFig} \cite{pubfig} (faces) and \textbf{OSR} \cite{oliva2001modeling} (outdoor scenes) datasets are used to compare the results of the proposed method with previous works. The PubFig dataset consists of 800 facial images of 8 random subjects. 11 attributes are defined in this dataset and attribute ordering of images is annotated in category level, \ie all images of a category may be ranked higher, equal, or lower than all images of another category, with respect to an attribute. The OSR dataset contains 2688 images in 8 categories, for which 6 relative attributes are defined. Like the PubFig dataset, relative ranking of attributes for this dataset have been annotated in category level.
Also \textbf{UT-Zap50K} \cite{Yu2014} (shoes) and \textbf{LFW-10} \cite{Sandeep_2014_CVPR} (faces) datasets, which are more challenging, have been used to assess the quality of the proposed method. The UT-Zap50K dataset consists of two collections, namely UT-Zap50K-1 in which \textit{coarse} relative attributes are compared for image pairs, and UT-Zap50K-2 in which \textit{fine-grained} relative attributes are compared for image pairs. The LFW-10 dataset consists of 2000 images and 10 attributes and for each attribute a random subset of 500 pairs of images have been annotated for each train and test set. Large number of categories in the UT-Zap50K and LFW-10 datasets makes them more challenging than the PubFig and OSR datasets. In addition to these datasets, to further analyze the properties of this end-to-end model and the feature hierarchy obtained, we have also experiemted with the MNIST \cite{lecun1998mnist} dataset. We have used class labels for images as the relative attribute and used the value of class label to rank images.
\subsection{Experimental setup}
\subsection{Baseline and compared methods}
\subsection{Results}
\subsection{Discussions}



% !TeX root=DeepRelAttr.tex
% !TEX TS-program = pdfLatex

%%%%%%%%%%%%%%%%%%%%%%%% CONCLUSION %%%%%%%%%%%%%%%%%%%%%%%%%%%%%%%%%%%%
\section{Conclusion}




{\small
\bibliographystyle{ieee}
\bibliography{refs}
}

\end{document}
